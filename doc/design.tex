%xxucms 设计过程文档

\documentclass{ctexart}

\author{徐潇}
\title{XxuCMS 设计说明}
\date{2017.12.26}

\begin{document}
  \maketitle  %生成上面的 title

  \section{系统环境、项目初始化}
  Ubuntu16.04 Emacs24.5 TexLive2005 Ctex Python3 Django2.0 Sqlite3
   \subsection{虚拟环境设置}
      在Ubuntu16.04下设置虚拟环境,在虚拟环境下安装各类软件、开发工具,互相不影响,只对当前虚拟环境产生影响。\\    
      \indent 1)安装虚拟环境软件\\      
      \indent >sudo apt install virtualenv \\
      \indent 2)创建一个虚拟环境(会自动建设目录,该目录就是一个虚拟环境,以后安装的所有的东西都在该目录中),这里创建了一个虚拟环境目录
      (virtualpython3),并没有启动,我使用了python3开发环境:\\
       \indent>virtualenv -p python3 virtualpython3 \\            
       \indent 3)启动/退出虚拟环境(先进入环境目录 virtualpython3):\\
       \indent >source ./bin/activate \\
       \indent >deactivate \\
    \subsection{安装django创建项目、应用}
      在虚拟环境下安装django并创建一个django项目及应用k。\\
      \indent 1)安装Django \\
      \indent >sudo apt install django \\
      \indent 2)创建django 项目与应用。 Django的项目是一个完整的WEB项目目录,里面可以包含多个应用,每个应用有其自己的视图、模式、模板,在项目中保存
      共有的URL、ADMIN管理、数据库管理等。每个应用程序可以看成项目的一个模块,采用多应用方式可以更好地实现迁移、复用。如下,先创建一个项目(自动创建一项目同名目录,
      生成一系列的目录。再生成一个应用(在项目中自动生成一个应用的同名目录,目录内含此应用的一系列文件)。\\
      \indent >django-admin startproject xxucms \\
      \indent >cd xxucms \\ 
      \indent >python.py startapp logreg \\
      \indent

  \newpage
  \section{开发记录}
    \subsection{}
      
      
   
  
    
  

\end{document}